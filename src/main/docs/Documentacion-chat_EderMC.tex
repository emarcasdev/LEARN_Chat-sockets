\documentclass[a4paper,12pt]{article}
% ===== Idioma y codificación =====
\usepackage[spanish]{babel}
\usepackage[T1]{fontenc}
\usepackage[utf8]{inputenc}
% ===== Tipografía =====
\usepackage{mathpazo}
% ===== Márgenes =====
\usepackage[margin=2.5cm]{geometry}
% ===== Interlineado =====
\usepackage{setspace}
\onehalfspacing  % 1.5
% ===== Imagenes =====
\usepackage{graphicx}
\graphicspath{{images/}}
\usepackage{caption}
\captionsetup[figure]{labelfont=bf,labelsep=period,name=Figura}
% ===== Tablas =====
\usepackage{booktabs}
\usepackage{array}
\usepackage{tabularx}
\usepackage{float}
\newcolumntype{Y}{>{\raggedright\arraybackslash}X}
\newcolumntype{C}{>{\centering\arraybackslash}X}
\captionsetup[table]{labelfont=bf,labelsep=period,name=Tabla}
% ===== Hipervínculos =====
\usepackage{xurl}
\usepackage[
  colorlinks=true,
  linkcolor=black,
  urlcolor=blue,
  citecolor=black,
  breaklinks=true
]{hyperref}
% ===== Código =====
\usepackage{xcolor}
\usepackage{listings}
\definecolor{codebg}{RGB}{245,245,245}
\lstdefinestyle{bash}{
  language=bash,
  basicstyle=\ttfamily\small,
  backgroundcolor=\color{codebg},
  frame=single,
  breaklines=true,
  showstringspaces=false
}
% ===== Estilo de secciones =====
\usepackage{titlesec}
\titleformat{\section}{\Large\bfseries}{\thesection}{0.5em}{}
\titleformat{\subsection}{\large\bfseries}{\thesubsection}{0.5em}{}
% ===== Encabezado y pie =====
\usepackage{fancyhdr}
\pagestyle{fancy}
\fancyhf{}
\fancyhead[L]{Chat Multi-Usuario}
\fancyhead[R]{2º DAM}
\fancyfoot[C]{\thepage}
% ===== Bibliografía =====
\usepackage[style=apa,sorting=nyt,backend=biber]{biblatex}
\addbibresource{bibliografia.bib}
\usepackage{csquotes}
% ===== Índice =====
\usepackage{tocloft}
\renewcommand{\contentsname}{}
\addto\captionsspanish{\renewcommand{\contentsname}{}}
\setlength{\cftbeforesecskip}{4pt}
\setlength{\cftbeforesubsecskip}{2pt}
\setlength{\cftsecindent}{0pt}
\setlength{\cftsubsecindent}{1.5em}
\renewcommand{\cftsecfont}{\bfseries}
\renewcommand{\cftsubsecfont}{\normalfont}
\renewcommand{\cftsecleader}{\cftdotfill{\cftdotsep}}
\renewcommand{\cftsubsecleader}{\cftdotfill{\cftdotsep}}
\renewcommand{\cftsubsubsecleader}{\cftdotfill{\cftdotsep}}


% ===== Documento =====
\begin{document}

\begin{titlepage}
  \centering
  \vspace*{3cm}

  {\Huge\bfseries Documentación Chat Multi-Usuario\par}
  \vspace{1.5cm}

  {\Large Chat multi-usuario empleando sockets y threads en Java\par}
  \vspace{2cm}

  {\large Eder Martínez Castro\par}
  \vspace{0.5cm}

  {\large Desarrollo de Aplicaciones Multiplataforma\par}
  \vfill

  {\large \today\par}
\end{titlepage}

\begin{center}
  {\Large\bfseries Tabla de contenidos}
\end{center}
\tableofcontents
\clearpage

\section{Objetivo}
\noindent El objetivo de esta práctica es \textbf{desarrollar} un \textbf{chat multi-usuario} en \texttt{Java} y \texttt{JavaFX} para la interfaz,
utilizando \textbf{sockets} para la comunicación cliente-servidor. Donde se busca que múltiples usuarios puedan conectarse simultáneamente y enviar mensajes
que lleguen al resto de usuarios en tiempo real.

\section{Análisis y Especificación de Requisitos}

\subsection{Descripción del problema}
\noindent Se necesita desarrollar una aplicación de chat que tenga lo siguiente:
\begin{itemize}
    \item Un \textbf{servidor}, que escuche conexiones entrantes en un puerto TCP.
    \item Que varios \textbf{usuarios} puedan conectarse con un nombre de usuario.
    \item Los mensajes enviados por un usuario, se envían a todos los usuarios conectados.
    \item La interfaz gráfica debe mostrar los mensajes diferenciando entre mensajes del \textbf{sistema}, \textbf{propios} y de \textbf{otros usuarios}.
\end{itemize}


\subsection{Requisitos funcionales}
\noindent A continuación, se listan los requisitos funcionales del sistema:
\begin{table}[H]
\centering
\begin{tabularx}{\textwidth}{@{}>{\bfseries}l Y@{}}
\toprule
ID & Requisito \\ \midrule
RF-01 & El cliente debe permitir introducir un nombre y poder conectarse al servidor. \\
RF-02 & El servidor debe aceptar múltiples conexiones simultáneas. \\
RF-03 & El cliente debe enviar mensajes de texto al servidor. \\
RF-04 & El servidor debe reenviar los mensajes recibidos a todos los usuarios conectados. \\
RF-05 & El cliente debe mostrar mensajes del sistema cuando alguien se conecta o desconecta. \\
RF-06 & El cliente debe permitir desconectarse escribiendo \texttt{salir}. \\
RF-07 & La interfaz debe diferenciar visualmente los mensajes propios, de los de los demás usuarios y de información. \\
\bottomrule
\end{tabularx}
\caption{Requisitos funcionales del chat multi-usuario}
\end{table}

\subsection{Requisitos no funcionales}
\noindent A continuación, se listan los requisitos no funcionales del sistema:
\begin{table}[H]
\centering
\begin{tabularx}{\textwidth}{@{}>{\bfseries}l Y@{}}
\toprule
ID & Requisito \\ \midrule
RNF-01 & El sistema debe soportar un máximo de 10 usuarios concurrentes. \\
RNF-02 & El servidor debe manejar desconexiones de forma segura liberando los recursos. \\
\bottomrule
\end{tabularx}
\caption{Requisitos no funcionales del chat multi-usuario}
\end{table}

\section{Arquitectura del Sistema}

\subsection{Arquitectura General}
\noindent La aplicación emplea la arquitectura \textbf{cliente-servidor} utilizando \textbf{sockets}.

\begin{itemize}
  \item \textbf{Servidor (Server)}:
  \begin{itemize}
    \item Abre un \texttt{ServerSocket} en el puerto \texttt{8080} y acepta conexiones.
    \item Atiende a cada usuario de forma concurrente (hilos) para permitir varios usuarios al mismo tiempo.
    \item Reenvía los mensajes a todos los usuarios conectados mediante \texttt{broadcast}.
  \end{itemize}

  \item \textbf{Cliente (JavaFX)}:
  \begin{itemize}
    \item Se organiza con \textbf{MVC} (Modelo: \texttt{ChatMessage}, Vista: \texttt{FXML/CSS}, Controlador: \texttt{ChatController}).
    \item Se conecta al servidor mediante un \texttt{Socket}.
    \item La obtención de los mensajes se maneja en otro \textbf{hilo} para no bloquear la UI.
  \end{itemize}

  \item \textbf{Comunicación}:
  \begin{itemize}
    \item Los mensajes se envían como \textbf{líneas de texto}.
    \item Al conectarse, el usuario envía primero su \textbf{nombre}.
    \item El comando \texttt{salir} desconecta al usuario.
  \end{itemize}
\end{itemize}

\begin{figure}[H]
\centering
\includegraphics[width=0.92\textwidth]{ModeloClienteServidor.png}
\caption{Diagrama de la arquitectura cliente-servidor}
\end{figure}

\subsection{Clases}
\noindent A continuación se describen las clases utilizadas en el chat multi-usuario:

\subsubsection{Server}
\noindent La clase \texttt{Server} implementa el servidor del chat, su función principal es la de \textbf{escuchar conexiones} y gestionar varios usuarios de forma concurrente:
\begin{itemize}
  \item Inicia un \texttt{ServerSocket} en el puerto \texttt{8080} y permanece en un bucle aceptando conexiones con \texttt{accept()}.
  \item Por cada usuario conectado, obtiene un \texttt{Socket} y lo asigna a una tarea del \textbf{pool de hilos}.
  \item Guarda el \texttt{PrintWriter} de cada usuario conectado para poder enviar el mismo mensaje a todos.
\end{itemize}

\noindent \textbf{Concurrencia:} se utiliza un \texttt{FixedThreadPool} con un máximo de \texttt{10} hilos, atendiendo varios usuarios a la vez sin crear hilos ilimitados.

\noindent \textbf{Difusión de mensajes:} el método \texttt{broadcast(...)} recorre \texttt{writers} y envía el mensaje a todos los usuarios que estén conectados.

\clearpage

\subsubsection{UserManager}                              
\noindent \texttt{UserManager} es una clase interna que se encarga de gestionar a un usuario conectado:
\begin{itemize}
  \item Obtiene el nombre del usuario y añade su \texttt{PrintWriter} a la lista \texttt{writers} para poder enviarle mensajes aparte de notificar la conexión del usuario.
  \item Escucha los mensajes en un bucle con \texttt{readLine()}.
  \item Si recibe \texttt{salir}, desconecta al usuario y notifica la desconexión.
  \item Al terminar, se elimina el \texttt{PrintWriter} del usuario de la lista \texttt{writers}, se notifica la desconexión y se cierra el socket.
\end{itemize}

\subsubsection{Client}
\noindent La clase \texttt{Client} se encarga de establecer y mantener la conexión con el servidor mediante un \texttt{Socket}, además de enviar y recibir mensajes:
\begin{itemize}
  \item En \texttt{connect(...)} se crea el socket y se inicializan los streams de entrada/salida usando UTF-8.
  \item Envía el nombre del usuario al servidor.
  \item Permite enviar mensajes con \texttt{send(...)} y cerrar la conexión con \texttt{close()}.
\end{itemize}

\noindent \textbf{Uso de Thread listener:} para no bloquear la interfaz, al conectar se inicia un hilo que escucha mensajes del servidor con \texttt{readLine()} y los entrega al controlador usando \texttt{Consumer<String> message}.

\subsubsection{ChatController}
\noindent \texttt{ChatController} es el controlador JavaFX encargado de gestionar la interfaz y coordinar la comunicación con el servidor:
\begin{itemize}
  \item Valida el nombre y llama a \texttt{client.connect(...)}.
  \item Habilita o deshabilita elementos de la interfaz según si está conectado o no.
  \item Envía mensajes con \texttt{client.send(...)} y limpia el campo para escribir el mensaje.
  \item Si el usuario escribe \texttt{salir}, envía el comando al servidor, cierra el socket y la aplicación.
\end{itemize}

\noindent \textbf{Actualización segura de la interfaz:} como los mensajes se reciben desde otro hilo, el controlador usa \texttt{Platform.runLater(...)} para añadir los mensajes en el \texttt{ListView} evitando así errores de concurrencia en JavaFX.

\subsubsection{ChatMessage}
\noindent \texttt{ChatMessage} es el modelo de datos que utiliza la interfaz para mostrar los mensajes. Incluye los diferentes tipos de mensaje (\texttt{INFO}, \texttt{ME}, \texttt{OTHER}) esto permite aplicar los estilos adecuados a la vista:
\begin{itemize}
  \item \texttt{INFO}: mensajes del sistema (conexión/desconexión).
  \item \texttt{ME}: mensajes enviados por el propio usuario.
  \item \texttt{OTHER}: mensajes enviados por otros usuarios.
\end{itemize}

\subsection{Stack empleado}
\noindent Para el desarrollo de nuestra aplicación hemos utilizado las siguientes tecnologías:
\begin{table}[H]
\centering
\begin{tabularx}{\textwidth}{@{}l Y@{}}
\toprule
Tecnología & Uso \\ \midrule
Java & Lenguaje utilizado en el desarrollo de la aplicación. \\
JavaFX & Interfaz gráfica del cliente. \\
Sockets & Comunicación cliente-servidor orientada a conexión. \\
Threads / ExecutorService & Concurrencia para atender múltiples usuarios y poder escuchar los mensajes sin bloquear interfaz. \\
FXML + CSS & Definición de la interfaz y estilos. \\
\bottomrule
\end{tabularx}
\caption{Tecnologías utilizadas}
\end{table}

\subsection{Interfaz}
\noindent La interfaz gráfica está compuesta por los siguientes elementos:
\begin{itemize}
  \item \textbf{Header}: formado por un campo para escribir el nombre y el botón de conexión.
  \item \textbf{Lista de mensajes}: \texttt{ListView} con celdas personalizadas para mostrar todos los distintos tipos de mensajes:
  \begin{itemize}
    \item \textbf{Mensajes del sistema}: mensajes centrados con color destacado.
    \item \textbf{Mensajes propios}: mensajes alineados a la derecha.
    \item \textbf{Mensajes de otros usuarios}: mensajes alineados a la izquierda.
  \end{itemize}
  \item \textbf{Footer}: formado por un campo para escribir mensajes y el botón de envío.
\end{itemize}

\begin{figure}[H]
\centering
\includegraphics[width=0.92\textwidth]{ProgramaFuncionando.png}
\caption{Interfaz gráfica del chat multi-usuario}
\end{figure}

\end{document}